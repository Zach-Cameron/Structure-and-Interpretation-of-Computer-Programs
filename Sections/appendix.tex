\newpage
\appendix

\section*{Appendix}

\section{Notes on LISP}
\label{sec:lisp}

\subsection{McCarthy 1960}

\href{http://jmc.stanford.edu/articles/recursive.html}{\textit{Recursive Functions of Symbolic Expressions and Their Computation by Machine}}

\subsubsection{Introduction}

LISP: 
\begin{itemize}
    \item "\textbf{LIS}t \textbf{P}rocessor"
    \item Developed for the IBM 704 computer by the Artificial Intelligence group at M.I.T. 
    \item Designed to facilitate experiments with a proposed system called the \textit{Advice Tracker}:
    \begin{itemize}
        \item A programming system for manipulating expressions representing formalized declarative and imperative sentences so that the Advice Tracker system could make deductions.
        \item Originally proposed in November 1958.
    \end{itemize}
    \item Came to be based on a scheme for representing the partial recursive functions of a certain class of symbolic expressions, independent of any electronic computer.
\end{itemize}

In this article:
\begin{enumerate}
    \item Describle a formalism for defining functions recursively.
    \item Describe S-expressions and S-functions, give examples, and describe the universal S-function \texttt{apply} which plays the theoretical role of a universal Turing machine and the practical role of an intepreter.
    \item Describe the representation of S-expressions in the memory of the IBM 704 by list structures ... and the representation of S-functions by program.
    \item Mention the main features of the LISP programming system for the IMB 704.
    \item Another way of describing computations with symbolic expressions.
    \item Give a recursive function interpretation of flow charts. 
\end{enumerate}

\subsubsection{Functions and Function Definitions}

\begin{defn}[\idx{partial function}]
    A function that is defined only on part of its domain.
\end{defn}

\begin{defn}[\idx{propositional expression}]
    A \textit{propositional expression} is an expression whose possible values are $T$ (for truth) and $F$ (for falsity)
\end{defn}

\begin{defn}[\idx{predicate}]
    A function whose range consists of the truth values $T$ and $F$.
\end{defn}

\begin{defn}[\idx{conditional expression}]
    A device for expressing the dependence of quantities on propositional quantities, denoted: \[(p_{1} \rightarrow e_{1}, \cdots, p_{n} \rightarrow e_{n})\] where the $p$'s are prositional expressions and the $e$'s are expression of any kind, read ``If $p_{1}$ then $e_{1}$, otherwise if $p_{2}$ then $e_{2}$, ... otherwise if $p_{n}$ then $e_{n}$'' or ``$p_{1}$ yields $e_{1}$, ..., $p_{n}$ yields $e_{n}$.''

    \textbf{How to determine the value of an arbitrary conditional statement $(p_{1} \rightarrow e_{1}, \cdots, p_{n} \rightarrow e_{n})$}:
    \begin{itemize}
        \item Let $i=1$.
        \item If the value of $p_{i}$ is undefined, then the value of the conditional expression is undefined.
        \item If the value of $p_{i}$ is $T$, then the value of the conditional expression is the value of the expression $e_{i}$.
        \item If the value of $p_{i}$ is $F$, then increment $i$ and evaluate $p_{i}$.
    \end{itemize}
\end{defn}

\begin{exmp}
    \hfill
    \begin{itemize}
        \item $(1 < 2 \rightarrow 4,\ 1 > 2 \rightarrow 3) = 4$
        
        Starting from the left with the statement $1 < 2 \rightarrow 4$, the propositional expression $1 < 2$ is true, therefore the value of the conditional expression is $4$.

        \item $(2 < 1 \rightarrow 4,\ 2 > 1 \rightarrow 3,\ 2 > 1 \rightarrow 2) = 3$
        
        Starting from the left with the statement $2 < 1 \rightarrow 4$, the propositional expression $2 < 1$ is false. Moving on to the next statement $2 > 1 \rightarrow 3$, the propositional expression $2 > 1$ is true, therefore the value of the conditional expression is $3$.

        \item $(2 < 1 \rightarrow 4,\ T \rightarrow 3) = 3$
        
        The propositional expression $2 < 1$ is false, so move on. The propositional expression $T$ is true, therefore the value of the conditional expression is $3$. 

        \item $(2 < 1 \rightarrow \frac{0}{0},\ T \rightarrow 3) = 3$
        
        The propositional expression $2 < 1$ is false, so move on. The propositional expression $T$ is true, therefore the value of the conditional expression is $3$.

        \item $(2 < 1 \rightarrow 3,\ T \rightarrow \frac{0}{0})$ is undefined.
        
        The propositional expression $2 < 1$ is false, so move on. The propositional expression $T$ is true, therefore the value of the conditional expression is the value of the expression $\frac{0}{0}$ which is undefined.

        \item $(2 < 1 \rightarrow 3,\ 4 < 1 \rightarrow 4)$ is undefined.
        
        The propositional expression $2 < 1$ is false, so move on. The propositional expression $4 < 1$ is false, so move on. There's nothing else to move on to, so the value of the conditional expression defaults to being undefined.
    \end{itemize}
\end{exmp}

\begin{exmp}
    Applications:
    \begin{itemize}
        \item $\left|x\right| = (x < 0 \rightarrow -x,\ T \rightarrow x)$
        \item $\delta_{ij} = (i=j \rightarrow 1, T \rightarrow 0)$
        \item $sgn(x) = (x < 0 \rightarrow -1,\ x = 0 \rightarrow 0,\ T \rightarrow 1)$
    \end{itemize}
\end{exmp}

\begin{exmp}
    Recursive applications:
    \begin{itemize}
        \item $n! = (n = 0 \rightarrow 1,\ T \rightarrow n \cdot (n-1)!)$
        \item $\gcd(m,n) = (m > n \rightarrow \gcd(n,m),\ rem(n,m) = 0 \rightarrow m,\ T \rightarrow \gcd(rem(n,m),m))$
        \item $\text{sqrt}(a, x, \epsilon) = (\left|x^{2} - a\right| < \epsilon \rightarrow x,\ T \rightarrow \text{sqrt}(a, \frac{1}{2}(x+\frac{a}{x}), \epsilon))$ 
    \end{itemize}
\end{exmp}

\begin{exmp}
    Propositional connectives defined by conditional expressions:
    \begin{itemize}
        \item $p \land q = (p \rightarrow q,\ T \rightarrow F)$
        \item $p \lor q = (p \rightarrow T,\ T \rightarrow q)$
        \item $\neg p = (p \rightarrow F,\ T \rightarrow T)$
        \item $p \supset q = (p \rightarrow q,\ T \rightarrow T)$
    \end{itemize}
\end{exmp}

\begin{defn}[\idx{form}]
    The expression $y^{2}+x$ is an example of a form.
\end{defn}

\begin{defn}[\idx{function}]
    A function substitutes arguments into a form to evaluate a result. It's necessarily to know which arguments map to which variables of the form. Something like $y^{2}+x(3,4)$ is not a function because it's not clear how $(3,4)$ should map to $x$ and $y$. If we make explicit the positions of the variables in the ordered tuple of arguments, then the mapping is clear. So, $\lambda((x,y),y^{2}+x)$ is a function, where e.g. $\lambda((x,y),y^{2}+x)(3,4)$ evaluates to $4^{2}+3 = 19$.

    A function of $n$ arguments is denoted \[\lambda((x_{1},\ldots,x_{n}),\mathcal{E})\] where $\mathcal{E}$ is a form of $n$ variables $x_{1},\ldots,x_{n}$.
\end{defn}

