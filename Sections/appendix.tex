\newpage
\appendix

\section*{Appendix}

\section{Notes on LISP}
\label{sec:lisp}

\subsection{\textit{Recursive Functions of Symbolic Expressions and Their Computation by Machine}, McCarthy 1960}

\href{http://jmc.stanford.edu/articles/recursive.html}{Link to article.}

\subsubsection{Introduction}

LISP: 
\begin{itemize}
    \item "\textbf{LIS}t \textbf{P}rocessor"
    \item Developed for the IBM 704 computer by the Artificial Intelligence group at M.I.T. 
    \item Designed to facilitate experiemtns with a proposed system called the \textit{Advice Tracker}:
    \begin{itemize}
        \item A programming system for manipulating expressions representing formalized declarative and imperative sentences so that the Advice Tracker system could make deductions.
        \item Originally proposed in November 1958.
    \end{itemize}
    \item Came to be based on a scheme for representing the partial recursive functions of a certain class of symbolic expressions, independent of any electronic computer.
\end{itemize}

In this article:
\begin{enumerate}
    \item Describle a formalism for defining functions recursively.
    \item Describe S-expressions and S-functions, give examples, and describe the universal S-function \texttt{apply} which plays the theoretical role of a universal Turing machine and the practical role of an intepreter.
    \item Describe the representation of S-expressions in the memory of the IBM 704 by list structures ... and the representation of S-functions by program.
    \item Mention the main features of the LISP programming system for the IMB 704.
    \item Another way of describing computations with symbolic expressions.
    \item Give a recursive function interpretation of flow charts. 
\end{enumerate}

\subsubsection{Functions and Function Definitions}

